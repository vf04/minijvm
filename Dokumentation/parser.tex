\documentclass{scrartcl}
\usepackage[utf8]{inputenc}
\usepackage[T1]{fontenc}
\usepackage[ngerman]{babel}
\usepackage{amsmath}
\usepackage{url}
\title{Dokumentation Java-Parser}
\author{Jakob Herrmann}
\date{WS-15-16}
\begin{document}
\maketitle
\begin{abstract}
Der Parser akzeptiert eine Klasse bzw. eine Menge von Klassen\footnote{Funktioniert leider nicht.}, erstellt mithilfe des Lexers eine Liste von Tokens und baut daraus den abstrakten Syntaxbaum auf.
\end{abstract}

\tableofcontents
\section{Verwendete Files}

\begin{itemize}
\item JavaLexer.x: das Alex-File mit den relevanten Tokens
\item JavaLexer.hs: das mit Alex erzeugte Haskell-Modul
\item JavaParser.y: das Happy-File; enthält die Regeln für die Java-Grammatik
\item JavaParser.hs: das mit Happy erzeugte Haskell-Modul
\item JavaParserHelper.hs: Haskell-Modul mit diversen Hilfsfunktionen für den Parser
\item AbsSyn.hs: Haskell-Modul mit den relevanten Datentypen für die abstrakte Syntax
\item Parser.hs: Auslesen eines Java-Files
\item Im Verzeichnis ./Testcode/parser befinden sich einige kleinere Java-Klassen, die mit dem Parser funktionieren.
\end{itemize}

\section{Funktionen des Parsers}

Der Parser wird mit folgender Funktion aufgerufen:
\begin{verbatim}
parse :: String -> [Class]
\end{verbatim}
...oder in Parser.hs durch die Funktion:
\begin{verbatim}
compileJavaFile(file)
\end{verbatim}
Die folgenden Classdeclarations werden erkannt:
\begin{itemize}
\item die leere Klasse
\item Klasse mit FieldDeclarations, jedoch ohne Assignments, da dies in der abstrakten Syntax nicht vorgesehen ist und wir das Modul nicht ändern wollten
\item Klasse mit Feldern und/oder Methoden, jedoch keine Konstruktoren
\item Modifiers werden generell erkannt, in der abstrakten Syntax jedoch nicht berücksichtigt.
\end{itemize}

\subsection{Method declarations}

Methoden können Folgendes enthalten:
\begin{itemize}
\item den leeren Block
\item lokale Variablen
\item Assignments\footnote{nicht mit Operatoren}
\item Return-Statements mit oder ohne Expression
\item If-Statements
\item While-Statements
\end{itemize}

\subsection{Expressions}

Folgende Expressions werden erkannt:
\begin{itemize}
\item Literale des Typs int und String
\item Unary Expressions
\item Rechenoperationen mit den Operatoren +, -, *, /+ \%\footnote{nur in Return-Statements}
\item Shiftexpressions
\item Equalityexpressions
\item Referenzen auf Klassen
\end{itemize}

\section{Nicht vollständig implementierte Funktionen}

\begin{itemize}
\item Methodcalls funktionieren nicht richtig und werden daher nicht erkannt.
\item Assignments mit Operatoren füren zu einer Endlosschleife und sind daher in der Grammatik zwar vorgesehen, jedoch deaktiviert.
\end{itemize}
\end{document}
